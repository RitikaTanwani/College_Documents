\documentclass [a4paper]{article}
\usepackage{amsmath}
\usepackage{amssymb}
\usepackage{url}
\usepackage{graphicx}
\usepackage{xcolor}
\usepackage{listings}
\usepackage{hyperref}
\usepackage{multirow}
\usepackage{multicol}
\title{\textbf{Assignment 2 - \LaTeX}}
\author{ITWS-2 Instructors \\ \texttt{itws-2-instr@mailinglists.iiit.ac.in}}
\date{January 29,2012}
\begin{document}
\maketitle

\renewcommand*\abstractname{Instructions}
\hline
\begin{abstract}
  You need to submit a \texttt{.tex} file which when compiled with PDF\LaTeX,
(\texttt{pdflatex}),will generate an exact copy of this \texttt{pdf}.The instructions are similar to the ones for the lab handout.
The list of packages for this assignment follows:
\begin{multicols}{3}
\begin{itemize}  
\item {\tt amsmath}   
\item {\tt amssymb}
\item {\tt url}
\item {\tt graphicx}     
\item {\tt xcolor}        
\item {\tt listings} 
\item {\tt hyperref}   
\item {\tt multirow}     
\item {\tt multicol} 
\end{itemize}
\end{multicols}
  Note that in this document,links(website references,emails,references,table of contents,etc.)are clickable but do not have colored boxes around them,giving the document a cleaner look\footnote{Hint:An optional argument to the hyperrefdoes this({\it this} by the way,is a footnote.)}.Also this ``Instructions'' section in actually the Abstract.You need to figure out how to change its displayed name.
\\

The image in Subsection 3.2 can be found here and the file regex.py used for listing 1 can be found here.
\\


Your assignment wiil be evaluated based on the extent of similarity of your PDF with this PDF 

\end{abstract}
\newpage
\tableofcontents

\newpage
\section{Typesetting text}
Paragraphs are automatically indented from the margin (unless specified otherwise),for example:


\paragraph {}This is a normal paragraph.It gets indented as one would expect.This is some random text inserted to increase the size of the paragraph. This text is also for the same reason
\\
\\
\noindent This is {\emph also} but isn't indented! Notice how the first line of this paragrph sticks to the left margin,unlike its counterpart in the above paragraph.\footnote{Hint:there's a command for this} 

\paragraph{}You already know how to typeset {\bf bold},{\it italicized},{\tt typewriter} and \textsf {serif} text,or to make text {\small small} or {\huge large}.

\\
The list in the Abstract should give you some food for thought.\footnote{Hint:It has {\underline {\tt multi}ple} {\underline{\tt col}umns!}}
\section{Mathematics}
\\
\noindent Here are some equations which have been numbered:
\\
\begin{equation}
(a+b)^{n}={\sum_{\substack{k=0}}^{\superstack{k=n}} \binom{n}{k}a^{k}b^{n-k}}
\end{equation}
\\
\begin{equation}
{\sum_{\substack{i=1}}^{\superstack{n}}a_i}\geq{\sqrt[n]{{\prod_{\substack{i=1}}^{\superstack{n}}a_i}}~~\text{,where $a_i$s are positive reals.}
\end{equation}
\noindent To write equation 2 correctly,you'll have to use \verb#\text#,so that you can get the normal text in math mode. Equation 2 can also be written as equation 3,  without the use of \verb#\text# . The {\bf R} in equation 3 has been generated using \verb#\mathbb{R}# which requires the {\tt amssymb} package.
\begin{equation}
\sum_{\substack{i=1}}^{\superstack{n}}a_i\geq\sqrt[n]{{\prod_{\substack{i=1}}^{\superstack{n}}}a_i}  ,~~~\forall a \in {\mathbb{R}_{+}}
\end{equation}
\noindent Here are a set of equation that have been alligned according to ``=''.Also note that they are not numbered:
\[
\begin{align}
(x+y)^{3}&=(x+y)^2(x+y)\\
&=(x^2+y^2+2xy)(x+y)\\
&=x^3+y^3+3x^2y+3y^2x
\end{align}
\]
\noindent Here are a few more equations:
\[20 \equiv 6\pmod 7\]
\newpage
\begin{equation}
{\text gcd(a,b)}|pa+qb
\end{equation}
%\begin{equation}
%F_n=\left\{
 % \begin{array}{l l}
 %   0 & \quad \text{if $n$ is 0}\\
  %  1 & \quad \text{if $n$ is 1}\\
  %  F_{n-1}+F_{n-2} & \quad \text{if $n$ is n\geq2}
  %\end{array} \right
%\end{equation}
\begin{equation}
%\begin{align}
\phi(n)=\sum_{\substack{i\leq k<n \\ gcd(k,n)=1}}1
\end{equation}
\begin{equation}
={n.\prod_{\substack{p|n}}\left(1-\frac{1}{p}\right)}       \text{,~~where $p$ is a prime}
%\end{align}
\end{equation}
\begin{equation}
\mathbb{S}=\left{x|x \geq 0,x\in \mathbb{Z}\right}
\end{equation}
Equation 4 says that the gcd of two numbers divides the sum of their multiples
while equation 5 is an inductive definition of the \textit {Fibonacci Series}.Equations 6
and 7 are two ways of computing \textit{Euler Totient Function.}Equation 8 defines
the set of all non-negative integers.
\\
\\
We are not done yet:
\begin{equation}
\boxed{^1/_6+^5/_6=1}
\end{equation}
\begin{equation}
  x = x_0 + \cfrac{1}{x_1
          + \cfrac{1}{x_2
          + \cfrac{1}{x_3 
          + \cfrac{1}{x_4}}}}
\end{equation}
\begin{equation}
\int_{0}^{n}\int_n^{\infty}(x^3+y^2)dx.dy
\end{equation}
\begin{center}
\begin{equation}
 A_{m,n} =
 \begin{pmatrix}
  a_{1,1} & a_{1,2} & \cdots & a_{1,n} \\
  a_{2,1} & a_{2,2} & \cdots & a_{2,n} \\
  \vdots  & \vdots  & \ddots & \vdots  \\
  a_{m,1} & a_{m,2} & \cdots & a_{m,n}
 \end{pmatrix}
\end{equation}
\end{center}
Equation 9 has a box around it, equation 10 displays a typical continued fraction,
equation 11 is an example of a double integral while equation 12 shows a generic
matrix of dimensions $m \times n$.
\newpage
\section{Floats}
Floats are containers for things in a document that cannot be broken over a
page. \LaTeX by default recognizes tables and figures as floats. Floats are there
to deal with the problem of the object that won��t fit on the present page, and
to help when you really don��t want the object here just now.\footnote{ Taken from the \LaTex Wikibooks page on Floats,Figures and Captions}.
\\
\\
You need to construct floats in this section.
\\
\subsection{Tables}
Columns can be merged in tables. This is done using \verb#\multicolumn#.Row can
also be merged using the \verb#\multirow# command, which is part of a package called
{\tt multirow}.
%table
The following points about Table 1 are worth noting:
\begin{itemize}
\item The first column is left aligned, while the rest are all right aligned.
\item The heading ��Planets�� spans 2 rows.
\item The heading ��Attributes�� spans 3 columns and is center aligned.
\item The line beneath the ��Attributes�� heading only spans rows 2, 3 and 4\footnote{Hint :{\tt cline}}.
\newpage
\subsection{Figures}
You did an example of a simple figure, with a caption, in the Lab handout.
Figure 1 builds upon the same concepts. It also has a border around it, using a
command called \verb#\fbox#.
\section{Listings}
Listing 1 shows a source code example in the Python programming language\footnote{\href{www.python.org}{Python Programming Language-Official Website}} ,displayed in the default Emacs python colour scheme.
\\
There are few notable points about Listing 1:
\begin{itemize}
\item  The {\verb#\ttfamily# font has been changed from its default value {\tt cmtt} (Com-
puter Modern Typewriter) to {\tt pcr} (Courier) just before the use of the
{\verb#\lstinputlisting# command.
(It has also been reset to the default immediately after.)\footnote{\cite{fonts} tells you how to make these changes}
\item The colours that have been used are\footnote{Consulting \cite{5} may help} :
\begin{itemize}[-]
\item {\bf Comments} \verb#\color[cmyk]#\left{0, 0.809, 0.809, 0.302\right}
\\
\item {\bf Keywords} \verb#\color[cmyk]#\left{0.12, 0.60, 0.00, 0.45\right}
\\
\item {\bf Strings} \verb#\color[cmyk]#\left{0.00, 0.76, 0.41, 0.45\right}\\
\end{itemize}
\end{itemize}
\newpage
\section{Macros}
Macros, as shown to you in the handout, can be quite useful if you have to use a
particular sequence (it could be a word or an equation or just about anything)
many times in your document.\\
~~~~~Take the word {\bf pneumonoultramicroscopicsilicovolcanoconiosis}, for in-
stance. It��s a lung disease, and if you were writing a paper/thesis about it, you��d
have to use the name quite a few times. The job can be simplified using a macro.
If you put

\begin{verbatim}
\newcommand{\pumsv}{pneumonoultramicroscopicsilicovolcanoconiosis}
\end{verbatim}

%\newcommand{\pumsv}{pneumonoultramicroscopicsilicovolcanoconiosis}

%\end{verbatim}
in the preamble of your document, you could just use \verb#\pumsv# to get pneu-
monoultramicroscopicsilicovolcanoconiosis wherever needed. (Notice how this
reduces the chance of making a spelling mistake, since the spelling only needs
to be correct in the macro definition.)
\\
\\
If you hate typing\verb# \LaTeX{}#, you can define a simple macro for that as well!
\subsection{Macros that take arguments}
Macros can also take arguments, like functions in your favourite programming
language. You could have a macro that typesets the list a0 to an given the
argument $a$.
\begin{verbatim}
\newcommand{\series}[1]{\[#1_0,#1_1,#1_2,\ldots,#1_n\]}
\end{verbatim}
\\
So,typing \verb#series{x}# would produce 
\[x_0,x_1,x_2,...,x_n\]
typing \verb#\series{\lambda}# would produce
\[ \lambda_0 , \lambda_1 , \lambda_2 , . . . , \lambda_n\]

and so on. \LaTeX{}  macros can be made to take any number of arguments.





\end{document}
