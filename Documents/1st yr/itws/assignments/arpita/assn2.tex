\documentclass[10pt,oneside,a4paper]{article}
%\usepackage{listings}
\usepackage{amsmath}
\usepackage{graphicx}
\usepackage{hyperref}
%\usepackage{amssymb}
%\usepackage{xcolor}
%\usepackage{multirow}
\usepackage{url}
\usepackage{multicol}
\title{\textbf{\Large{Assignment 2 - \LaTeX{}}}}
\author{ITWS-2 Instructors\\itws-2-instr@mailinglists.iiit.ac.in}
\date{\today}
\begin{document}
\maketitle
%\hline
\renewcommand{\abstractname}{Instructions}
\begin{abstract}
\label{abstract}


You need to submit a \textsf{.tex} file which when compiled with PDF\LaTeX,
(pdflatex),will generate an exact copy of this \textsf{pdf}.The instructions are similar to the ones for the labhandout.The list of packages used for this assignment follows:
\\
\begin{multicols}{3}
\begin{itemize}
\item{amsmath}
\item{graphicx}
\end{itemize}
\end{multicols}
Note that in this document, links (website references, emails, refer-
ences, table of contents items, etc.) are clickable but do not have coloured
boxes around them, giving the document a cleaner look \footnote{Hint: an optional argument to the \textsf{hyperref} package does this. (this,by the way, is a footnote.)}.Also, this “In-structions” section is actually the Abstract. You need to figure out how to change its displayed name.\\

The image in Subsection 3.2 can be found \href{http://pascal.iiit.ac.in/~sankalp/itws2/latexassignment/latex.png}{\textit{here}} and the file \textsf{regex.py} used for Listing 1 can be found \href{http://pascal.iiit.ac.in/~sankalp/itws2/latexassignment/regex.py}{\textit{here}}.\\

Your assignment will be evaluated based on the extent of similarity of
your \textsc{pdf} with this \textsc{pdf} .
\end{abstract}
\newpage
\tableofcontents \label{contents}
\newpage
\section{Typesetting text}
Paragraphs are automatically indented from the margin (unless specified other-wise), for example :\\

This is a normal paragraph. It gets indented as one would expect. This is some random text inserted to increase the size of the paragraph. This text is also for the same reason.\label{para1}\\

\noindent This is also a paragraph but isn’t indented! Notice how the first line of this paragraph sticks to the left margin, unlike its counterpart in the above para-graph \footnote{Hint :there's a command that does this.}\\
You already know how to typeset \textbf{bold},\textit{italicized},\texttt{typewriter} and \textsf{serif} text, or to make text \small{small} or \Large{large}.\\

\normalsize The list in the Abstract should give you some good for thought\footnote{Hint: It has multiple coloumns!} .

\section{Mathematics}
Here are some equations which have been numbered:

\begin{equation}
\label{powern}
(a+b)^{n} = \displaystyle \sum_{k=0}^{k=n} \binom{n}{k}a^{k}b^{n-k}
\end{equation}

\begin{equation}
\label{partof:1}
\displaystyle \sum_{i=1}^{n} \geq \sqrt[n]{\prod_{i=1}^{n} a_i} ,\forall a_i \in R 
\end{equation}
\section{macros}
Macros, as shown to you in the handout, can be quite useful if you have to use a particular sequence (it could be a word or an equation or just about anything) many times in your document.\\
Take the word pneumonoultramicroscopicsilicovolcanoconiosis, for in-
stance. It’s a lung disease, and if you were writing a paper/thesis about it, you’d have to use the name quite a few times. The job can be simplified using a macro.If you put

\newpage
\begin{thebibliography}{5}
\bibitem{tutorial} \LaTeX{} Tutorials, a Primer, \emph{Indian \TeX{} Users Group (TUG India),} TUG India, 2002-2003
\bibitem{packages} 
LaTeX/Packages/Listings - Wikibooks, open books for an open world,\\\url{http://en.wikibooks.org/}
\bibitem{intro} 
The Not So Short Introduction to \LaTeXe, OR, \LaTeXe in 157 minutes,
\emph{Oetiker T., Hyna I. and Schlegl E.,} Version 5.01, April6,2001
\bibitem{fonts} Fonts in \LaTeX, \emph{Freek Dijkstra,}\\
\url{http://www.macfreek.nl/memory/Fonts_in_LaTeX}
\bibitem{5} Using listings to include code in \LaTeX{}, \emph{Thomas Jansson,}\\
\url{http://www.tjansson.dk/?p=419}
\end{thebibliography}

\end{document}
