\documentclass{article}
\title{\textbf{Assignment 2 - {\LaTeX}}}
\author{ITWS-2 Instructors\\ \texttt{itws-2-instr@mailinglists.iiit.ac.in}}
\date{January 29,2012}
\usepackage{amsmath,graphicx,amyssymb,xcolor,multirow,url,listings,multicol}
%\usepackage{hyperref}
%\hypersetup{colorlinks=false}
\usepackage[colorlinks=true,linkcolor=black,anchorcolor=black,citecolor=black,filecolor=black,menucolor=black,runcolor=black,urlcolor=black]{hyperref}
\renewcommand{\abstractname}{Instructions}
\begin{document}
\maketitle
%\rule{\linewidth}{0.5mm}
\hline
\begin{abstract}
%\begin{center}
%\textbf{Instructions}
%\end{center}
You need to submit a \texttt{.tex} file which when compiled with PDF{\LaTeX},
(\texttt{pdflatex}), will generate an exact copy of this \texttt{pdf}. The instructions are similar to the ones for the lab handout. The list of packages used for this assignment follows:

\begin{multicols}{3}
\begin{itemize}
\item \texttt{amsmath}
\item \texttt{amyssymb}
\item \texttt{url}
\item \texttt{graphicx}
\item \texttt{xcolor}
\item \texttt{listings}
\item \texttt{hyperref}
\item \texttt{multirow}
\item \texttt{multicol}
\end{itemize}
\end{multicols}

Note that in this document, links (website references, emails, refer\-ences, table of contents, etc.) are clickable but do not have coloured boxes around them, giving the document a cleaner look\footnote{Hint: an optional argument to the \texttt{hyperref} package does this. (\emph{this}, by the way, is a footnote)}. Also, this ``Instructions'' section is actually the Abstract. You need to figure out how to change its displayed name.\\

The image in Subsection 3.2 can be found \href{http://pascal.iiit.ac.in/~sankalp/itws2/latexassignment/latex.png}{\textit{\underline{here}}} and the file \texttt{regex.py} used for Listing 1 can be found \href{http://pascal.iiit.ac.in/~sankalp/itws2/latexassignment/regex.py}{\textit{\underline{here}}}.\\

Your assignment will be evaluated based on the extent of similarity of your \textsc{pdf} with this \textsc{pdf}
\end{abstract}
\newpage
\tableofcontents
\newpage
\section{Typesetting text}
Paragraphs are automatically indented from the margin (unless specified other-
wise), for example :\\

This is a normal paragraph. It gets indented as one would expect. This is
some random text inserted to increase the size of the paragraph. This text is
also for the same reason.\\

\noindent This is also a paragraph but isn’t indented! Notice how the first line of this
paragraph sticks to the left margin, unlike its counterpart in the above para-
graph\footnote{Hint: there's a command that does this.}\\

You already know how to typeset \textbf{bold}, \textit{italicized}, \texttt{typewriter} and \textsf{serif} text, or to make text {\small small} or {\Huge large}.\\

The list in the Abstract should give you some food for thought\footnote{Hint: It has \underline{multi}ple \underline{col}umns!}.

\section{Mathematics}
Here are some equations which have been numbered:
%\[
\begin{equation} \label{eq:eq1}
(a + b)^n = {\sum_{k=0} ^ {k=n}} \binom{n}{k}a^kb^{n-k}
\end{equation}
%\]
%\[
\begin{equation} \label{eq:eq2}
{\sum_{i=1}^{n}}a_{i} \geq {\sqrt[n]{\prod_{i=1}^{n}a_{i}}}~~ \text{, where $a_i$s are positive reals.}
\end{equation}
%\]
To write equation \ref{eq:eq2} correctly, you'll have to use \verb#\text#, so that you can get normal text in math mode. Equation \ref{eq:eq2} can also be written as equation \ref{eq:eq3}, without the use of \verb#\text#. The \textbf{R} in equation \ref{eq:eq3} has been generated using the \verb#\mathbb{R}# command, which requires the \verb#amssymb# package.   
%\[
\begin{equation} \label{eq:eq3}
\sum_{i=1}^{n}a_{i}\geq \sqrt[n]{\prod_{i=1}^{n}a_{i}}~~~, \forall a_{i} \in \mathbf{R}_{+}}
\end{equation}
%\]
Here are a set of equations that have been aligned according to ``=''. Also note that they are not numbered:
\[
\begin{align}
(x + y)^{3} &= (x + y)^{2}(x + y)\\
&= (x^{2} + y^{2} + 2xy)(x + y)\\
&= x^{3} + y^{3} + 3x^{2}y + 3xy^{2}\\
\end{align}
\]
\nopagebreak[0]
Here are a few more equations:
\[
20 \equiv 6 \pmod 7
\]
%page to b inserted
\section{Floats}
Floats are containers for things in a document that cannot be broken over a
page.{\LaTeX} by default recognizes tables and figures as floats. Floats are there
to deal with the problem of the object that won’t fit on the present page, and
to help when you really don’t want the object here just now.\footnote{Taken from the {\LaTeX} Wikibooks page on Floats, Figures and Captions}.
\paragraph{}
You need to construct floats in this section.
\subsection{Tables}
Columns can be merged in tables. This is done using \verb#\multicolumn#. Rows can also be merged using the \verb#\multirow# command, which is part of a package called \texttt{multirow}.\\
%\begin{center}
\begin{table}[h]
\begin{center}
\begin{tabular}{|c|r|r|r|}\hline
\multirow{2}{*}{\textbf{Planet}} & \multicolumn{3}{|c|}{\textbf{Attributes}}\\
\cline{2-4}
&  Radius(km) & Mass(\times10^{21}) & Density(g/cm^{3})\\\hline
Mercury & 2,439.7 & 330.2 & 5.43\\
Earth & 6,371 & 5,973.6 & 5.515\\
Mars & 3,390 & 641.85 & 3.94\\
Jupiter & 69,911 & 1,898,600 & 1.33\\\hline
\end{tabular}
\caption{Planets}
\end{center}
\end{table}\\
The following points about Table 1 are worth noting:\\
\begin{itemize}
\item The first column is left aligned, while the rest are all right aligned
\item The heading ``Planets'' spans 2 rows.
\item The heading ``Attributes'' spans 3 columns and is center aligned.
\item The line beneath the ``Attributes'' heading only spans rows 2,3 and 4\footnote{Hint: \texttt{cline}}
\end{itemize} 




\end{document}
