\documentclass[a4paper,10pt,titlepage]{article}
\usepackage{url}
\usepackage{graphicx}
\usepackage{listings}
\usepackage{xcolor}
\usepackage{hyperref}
\usepackage{amsmath}

% LaTeX Handout for Week 4 Labs.
% Mihir, Sankalp.

% We're deliberately not including some of the more advanced topics,
% like labels and cross-referencing, citations & bibliographies,
% tables with merged cells, advanced mathematics, customizing listings
% using \lstset, to keep the level of the handout such that it might be
% done by the average first year in 2 hours, hopefully.

% We'll probably post an assignment separately, which will be more
% involved.

\lstset{
  language=C,                  % the language of the code
  basicstyle=\footnotesize\ttfamily,       % the size of the fonts that are used for the code
  numbers=left,                   % where to put the line-numbers
  numberstyle=\tiny\color{black},      % the size of the fonts that are used for the line-numbers
  backgroundcolor=\color{lightgray},  % choose the background color. You must add \usepackage{xcolor}
  showstringspaces=false,         % underline spaces within strings
  frame=double,                   % adds a frame around the code
  captionpos=b,                   % sets the caption-position to bottom
  commentstyle=\ttfamily\color{blue},   % set colour blue for comments
  keywordstyle=\color{orange},
  stringstyle=\color{red},
}

\title{\LaTeX{} Lab Handout}
\author{ITWS-2 Instructors \\ \texttt{itws-2-instr@mailinglists.iiit.ac.in}}
\date{\today}

\newcommand{\qubit}{$|\psi\rangle$}
\newcommand{\Marks}[1]{\hspace*{\fill}{\footnotesize{(\textbf{#1 points})}}}
\newcommand{\Mark}[1]{\hspace*{\fill}{\footnotesize{(\textbf{#1 point})}}}

\begin{document}
\maketitle

\tableofcontents

\newpage

\begin{abstract}
  You need to submit a \TeX\ source which when compiled using
  pdf~\LaTeX\ (the command \texttt{pdflatex}), produces an exact copy
  of this \textsc{pdf}, \textbf{except the sections on passwordless ssh and the marking scheme}.\\

  The only difference being that it should have your Name, Roll
  No.~and Email in the author field, in place of ours, and the date
  of compilation (today) in the date field, if it differs from what's
  present in this file. Try to match
  the typesetting as closely as you can.\\

  \noindent Each of the following carries marks:
  \begin{enumerate}
  \item Content
  \item Text formatting
  \item Document structure
  \end{enumerate}

  \noindent This document has been typeset using the \texttt{article}
  documentclass, with paper dimensions \texttt{a4} and font size \texttt{10pt}.\\

  \noindent The following packages have been used:
  \begin{itemize}
  \item \texttt{url} to display URLs
  \item \texttt{hyperref} to make links (including URLs) clickable
    % \item \texttt{parskip} so that paragraphs all start from the left
    %   margin
  \item \texttt{graphicx} to include images
  \item \texttt{listings} to include source code
  \item \texttt{xcolor} to get a variety of font colours, referrable
    by name (used in the source code listings)
  \end{itemize}

  \noindent Use the image at: \url{http://xkcd.com/149/} for Figure \ref{fig:sandwich}\\

  \noindent Consulting The Wikibooks guide to \LaTeX{} (link
  in the Online References section of the course website) should help.\\

  \textbf{Note: } If you look closely, you'll see that this abstract
  has been generated using a special command, which is part of the
  article documentclass (it has a lesser font size and a smaller
  margin)! Find it.

  \textbf{Note: } We have not used \textsc{Bib}\TeX{} for the References. They
  are written in the \texttt{tex} file itself.
\end{abstract}

\section{Typesetting Text}

There are many different ways of embellishing text provided by
\LaTeX{}:

\begin{itemize}
\item \textbf{Bold Text}
\item \emph{Emphasized Text}
\item \textsl{Slanted Text}
\item \textsc{Text in Small Caps}
\item \texttt{Text in Teletype Font}
\item \textsf{Text in the Serif Font}
\item \textrm{Text in the Roman Font (default)}
\end{itemize}
Notice that emphasized text is different from italicized text. \textit{For example, this text is written in italics, but \emph{these words are
    emphasized text}}. Also, this is normal text, but \emph{this is emphasized text}.
\paragraph{}
The above is achieved using this: 
\begin{verbatim}
\textit{For example, this text is written in italics, but 
      \emph{these words are emphasized text}}. Also, this 
      is normal text, but \emph{this is emphasized text}.
\end{verbatim}
Text can be superscripted: \textit{``Remember, Remember the 5\textsuperscript{th} of November!''}\\

The {\Large size} of {\scriptsize text} can {\LARGE also} be {\Huge
  varied}.

\section{Typesetting Mathematics}

Math can be typeset inline, for example: $a= b^2 + c$, or displayed
separately as shown in the rest of this section.\\

\noindent \textit{Note: Do not get intimidated, there are simple commands for
  generating most of the symbols/structures shown below!}

\subsection{General Formulae}

\[ D = \frac{-b \pm \sqrt{b^2 - 4ac}}{2a} \]

% \[
% {n \choose r} = \displaystyle\sum_{k = r - 1}^{k = n - 1}
% {k \choose r - 1}
% = \displaystyle\sum_{k = 0}^{k = r}
% {n - 1 - k \choose r}
% \]

\subsection{Calculus}

% \[\lim_{x\rightarrow\infty}\frac{x^2 + 4x + 4}{x^3 + 6x^2 + 12x + 8}=0 \]

% \[
% \frac{\partial f}{\partial x} = \frac{\partial f}{\partial y} \frac{\partial y}{\partial x}
% \]

\[ \int_0^{+\infty} x^n e^{-x} \,dx = n!\]

\subsection{Counting}

\[{^nC_r} = \binom{n}{r} = \frac{n!}{r!(n-r)!}\]
\[{^nP_r} = \frac{n!}{(n-r)!}\]

\subsection{Trigonometry}

\[\sin^2\theta + \cos^2\theta=1\]

%\subsection{Matrices}
% \[
% A =
% \begin{pmatrix}
%   a_{1,1} & a_{1,2} & a_{1,3} & \cdots & a_{1,n} \\
%   a_{2,1} & a_{2,2} & a_{2,3} & \cdots & a_{2,n} \\
%   a_{3,1} & a_{3,2} & a_{3,3} & \cdots & a_{3,n} \\
%   \vdots  &  \vdots &  \vdots & \ddots & \vdots \\
%   a_{M,1} & a_{M,2} & a_{M,3} & \cdots & a_{M,n}
% \end{pmatrix}
% \]

\subsection{Miscellaneous}

\[\vec{F} = m\vec{v}\]
\[q = ms\Delta t\]
%\[\rho = \frac{m}{V}\]
\[\overline{(A\land B)} = \overline{A}\lor \overline{B}\]

\section{Tables}

Here is an elementary table:
\begin{center}
  \begin{tabular}{l p{5.5cm} c}
    \textbf{\#} & \textbf{Column 1} & \textbf{Column 2} \\
    \hline
    \hline
    1. & Notice that this column has a predefined width - 5.5cm & This one does not. \\\hline
    2. & These lines will help you understand. & These will too. \\[50pt]\hline
    3. & There is an extra gap between rows ``2.'' and ``3.''. You have to find a way to can get this gap. & Some text here. \\\hline
    4. & (No, you can't get the desired output by ``leaving a row blank'') & (Try it.) \\ \hline
    5. & Tables which have minimal vertical lines are more readable than those with a large number of vertical lines & ... \\\hline
    \hline
  \end{tabular}
\end{center}
% The following table demonstrates the above fact:
% \begin{center}
%   \begin{tabular}{| c | c | c |}
%     Col 1 & Col 2 & Col 3 \\
%     \hline
%     This table & has lots of & vertical lines \\
%     which makes & it look less & readable than \\
%     the one & shown earlier, & with \\
%     more & horizontal & lines
%   \end{tabular}
% \end{center}

\section{Creating your own commands/macros}

This is a very powerful feature in \LaTeX{}. For example, a qubit in
quantum computing, $|\psi\rangle$, is written in \LaTeX{} source as
\verb#$|\psi\rangle$#. Writing this everywhere is messy, so we define
a new command (macro) in the document preamble:
\begin{verbatim}
    \newcommand{\qubit}{$|\psi\rangle$}
\end{verbatim}
Now, \verb#\qubit{}# produces the same result as
\verb#$|\psi\rangle$#, which is \qubit{}.\\(This last instance of
the qubit should be produced using a macro in your \TeX{} source.)

\section{Embedding images in your document}
\LaTeX\ also supports images. Use the \verb#\includegraphics# command to include images
in documents. Using the \verb#\figure# environment, you can add
captions to images.
\begin{figure}[h]
  \centering
  \includegraphics[width=3.5cm]{sandwich.png}
  \caption{xkcd: Sandwich. The width of this image is 3.5cm}
  \label{fig:sandwich}
\end{figure}

\section{Displaying Source Code}
The \texttt{listings} package is used to display source code in a
document. It is often convenient to define the parameters to listings
in the preamble itself (e.g.~if all code listings are in the same
language.)
% \begin{lstlisting}[language=C,numbers=left,basicstyle=\ttfamily,keywordstyle=\color{orange},commentstyle=\color{blue}\ttfamily,backgroundcolor=\color{lightgray},]
%   \begin{lstlisting}[language=C, numbers=left]

\begin{lstlisting}[caption=My C program!,label=lst:hello-world-c]
  #include<stdio.h>
  // My program
  int main()
  {
    printf("Number Tester\n\nEnter a Number: ");
    int n;
    scanf("%d",&n);
    if(n>=0)
    {
      printf("positive\n");
    }
    else
    {
      printf("negative\n");
    }
  }
\end{lstlisting}

\noindent The following are some of the main options which have been
set globally for listings in this file, to get the desired effect in
Listing \ref{lst:hello-world-c}:
\begin{itemize}
\item \verb#language=C# (source code language)
\item \verb#basicstyle=\footnotesize\ttfamily# (basic font style)
\item \verb#keywordstyle=\color{orange}#
\item \verb#commentstyle=\color{blue}\ttfamily#
\item \verb#numbers=left# (line numbers, and where to put them)
\item \verb#backgroundcolor=\color{lightgray}#
\end{itemize}

Try figuring out the rest of the options
yourself. \cite{wikibooks-listings} should give you an idea
of what is to be done.

\newpage
\section{Setting up Passwordless Logins with SSH}
\textit{\textbf{Note:} You don't have to include this section in your
document. This section can be left out.}\\


  \texttt{ssh} supports what is known as \emph{key-based} authentication. It
  involves the generation of a key-pair, one of which is called the
  \emph{private} key and the other the \emph{public} key. These two fit together
  like pieces of a jigsaw puzzle. Each public key only matches with
  its private counterpart, and vice-versa.

  Lets now go over the process of generating and using these keys for
  setting up \texttt{ssh} access without being queried for a password each
  time.\\

  \noindent \texttt{harry} wants to setup passwordless access to a remote machine. Let
  us assume you are \texttt{harry}.
\subsection{Generating a Key Pair}


\begin{itemize}
\item execute the \texttt{ssh-keygen} command:
\end{itemize}

\begin{verbatim}
 [harry@privetdrive ~]$ ssh-keygen
 Generating public/private rsa key pair.
 Enter file in which to save the key (/home/harry/.ssh/id_rsa):
\end{verbatim}

\begin{itemize}
\item When queried for the file, press enter (i.e.~choose the default
  location).
\item If your user doesn't have a \texttt{.ssh} directory, it will be created
    and you will get the following message:
\end{itemize}

\begin{verbatim}
 Created directory '/home/harry/.ssh'.
\end{verbatim}

\begin{itemize}
\item It will then ask you for a passphrase, which is used each time you
    want to use the key.
\end{itemize}

\begin{verbatim}
 Enter passphrase (empty for no passphrase):
\end{verbatim}

\begin{itemize}
\item Leave the passphrase empty (press enter) since our intent is to
    not be queried for anything at all.
\end{itemize}

\begin{verbatim}
 Enter same passphrase again:
\end{verbatim}

\begin{itemize}
\item After this, you should see output similar to the following:
\end{itemize}

\begin{verbatim}
 Your identification has been saved in /home/harry/.ssh/id_rsa.
 Your public key has been saved in /home/harry/.ssh/id_rsa.pub.
 The key fingerprint is:
 10:39:09:e0:77:ef:90:c8:9e:14:30:fd:6c:72:db:41 harry@privetdrive
 The key's randomart image is:
 +--[ RSA 2048]----+
 |  +o...o         |
 | . o. +.E        |
 |  . oooo         |
 |   o.+=+.        |
 |    ++ooS.       |
 |   o ..o.        |
 |    o   .        |
 |                 |
 |                 |
 +-----------------+
\end{verbatim}
\subsection{Using the Key Pair for passwordless SSH}


   Now that the key pair is generated, we need to setup things so that
   the public key is placed in a specific file on the remote machine
   (server) where you want to login, without being queried for a
   password. The private key must remain in the \texttt{\textasciitilde{}/.ssh} folder
   (default location where it was placed by \texttt{ssh-keygen}) for this to
   work.

\begin{itemize}
\item Suppose you have an account on a remote machine called
     \texttt{hogwarts}, with the same user id (\texttt{harry}). The following
     command will place your public key in the appropriate location on
     \texttt{hogwarts}.
\end{itemize}

\begin{verbatim}
 [harry@privetdrive ~]$ ssh-copy-id harry@hogwarts
 The authenticity of host 'hogwarts (256.256.256.256)'
 can't be established.
 RSA key fingerprint is
 47:18:e3:f6:d6:ea:b9:58:96:96:04:74:9f:6c:1c:fe.
 Are you sure you want to continue connecting (yes/no)? yes
 Warning: Permanently added 'hogwarts,256.256.256.256' (RSA)
 to the list of known hosts.
 harry@hogwarts's password:
\end{verbatim}

\begin{itemize}
\item When queried, enter your password.
\item You will see a message similar to the one below:
\end{itemize}

\begin{verbatim}
 Now try logging into the machine, with "ssh 'harry@hogwarts'",
 and check in:

   ~/.ssh/authorized_keys

 to make sure we haven't added extra keys that you weren't expecting.

\end{verbatim}

\begin{itemize}
\item Your setup is complete.
\end{itemize}

Voila! \texttt{harry} should now be able to enter \texttt{hogwarts} without a password!

\newpage
\section{Marking Scheme}
\textit{\textbf{Note:} You don't have to include this section in your
document. This section can be left out.}
\subsection{Typesetting text - 3 marks}
\begin{itemize}
 \item \textbf{2 marks} for different text styles.
 \item \textbf{1 mark} for sub/superscripts and font size.
\end{itemize}

\subsection{Typesetting Mathematics - 3 marks}
\begin{itemize}
 \item Sections 2.1 to 2.4 - \textbf{0.5 marks} each.
 \item Section 2.5 - \textbf{1 mark}.
\end{itemize}

\subsection{Tables - 3 marks}
\begin{itemize}
 \item \textbf{1 mark} for width of 5.5cm for Column 1 in first table.
 \item \textbf{1 mark} for a gap of 50pt between row 2 and row 3.
 \item \textbf{0.5 marks} for the double horizontal lines (at the top
   and bottom)
 \item \textbf{0.5 marks} for the single horizontal lines.
\end{itemize}

\subsection{Macro - 1 mark}
\textbf{1 mark} for proper definition and use of the macro.

\subsection{Image - 3 marks}
\begin{itemize}
 \item \textbf{1 mark} for including the image.
 \item \textbf{1 mark} for the caption.
 \item \textbf{1 mark} for centering.
\end{itemize}

\subsection{Listings - 3 marks}
\textbf{3 marks} for the look, language, colours, line numbering,
fonts, etc.

\subsection{Passwordless SSH}
\textbf{4 marks} Demonstrating passwordless login from your lab
machine to web.\\
\textbf{Please unset it before leaving.}

\newpage
\begin{thebibliography}{10}
\addcontentsline{toc}{section}{References}
\bibitem{ltxprimer-1} \LaTeX\ Tutorials, a Primer, \emph{Indian \TeX\ Users Group (TUG India)}, TUG India, 2002-2003
\bibitem{intro-to-latex} The Not So Short Introduction to \LaTeXe , OR, \LaTeXe\ in 157 minutes, \emph{Oetiker T.,
    Hyna I. and Schlegl E.}, Version 5.01, April 6, 2001
\bibitem{compre-symb-list} The Comprehensive \LaTeX\ Symbol List, \emph{Scott Pakin}, 2009
\bibitem{wikibooks-listings} LaTeX/Packages/Listings - Wikibooks, open books for an open world, \url{http://en.wikibooks.org/}

\end{thebibliography}


\end{document}

